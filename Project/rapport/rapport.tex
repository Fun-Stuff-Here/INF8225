%%%% rapport.tex

% These are the instructions for authors for IJCAI-19.

\documentclass{article}
\pdfpagewidth=8.5in
\pdfpageheight=11in
% The file ijcai19.sty is NOT the same than previous years'
\usepackage{ijcai19}

% Use the postscript times font!
\usepackage{times}
\usepackage{soul}
\usepackage{url}
\usepackage[hidelinks]{hyperref}
\usepackage[utf8]{inputenc}
\usepackage[small]{caption}
\usepackage{graphicx}
\usepackage{amsmath}
\usepackage{booktabs}
\usepackage{algorithm}
\usepackage{algorithmic}
\urlstyle{same}


\title{Exploration d'un Resnet pour la classification d'images}


\author{
Nicolas Dépelteau
\and
Dung Nguyen
\affiliations
Polytechnique Montréal
\emails
nicolas.deplteau@polymtl.ca,
thi-ngoc-dung.nguyen@polymtl.ca

}
\begin{document}

\maketitle

\begin{abstract}
    Dans ce rapport écrit pour notre projet dans le cadre du cours INF8225,
    nous présentons les bénificents de l'utilisation
    d'un Resnet dans le cadre de la classification d'images. Nous avons également comparé les
    résultats de notre Resnet avec ceux d'un réseau de neurones convolutif simple. 
\end{abstract}

\section{Introduction}

TODO Une introduction avec un petit résumé des travaux antérieurs qui existent dans la littérature

\section{Les réseaux convolutifs}

TODO l'expliquation des réseaux convolutifs
\subsection{Convolution}

TODO l'expliquation de la convolution

\subsection{Pooling}

TODO l'expliquation du pooling
\subsection{Le problème  de dégradation}

TODO l'expliquation du problème de dégradation
\section{Resnet}

TODO l'expliquation des Resnet

\section{Expérimentation}

TODO l'expliquation de l'expérimentation

\subsection{Le jeu de données}

TODO l'expliquation du jeu de données

\subsection{Les hyperparamètres}

TODO l'expliquation des hyperparamètres

\subsection{Les résultats}

TODO l'expliquation des résultats

\section{Analyse critique de l'approche}

TODO l'analyse critique de l'approche

\section{Conclusion}

TODO la conclusion


\bibliographystyle{named}
\bibliography{rapport}

TODO la bibliographie

\end{document}
