%%%% rapport.tex

% These are the instructions for authors for IJCAI-19.

\documentclass{article}
\pdfpagewidth=8.5in
\pdfpageheight=11in
% The file ijcai19.sty is NOT the same than previous years'
\usepackage{ijcai19}

% Use the postscript times font!
\usepackage{times}
\usepackage{soul}
\usepackage{url}
\usepackage[hidelinks]{hyperref}
\usepackage[utf8]{inputenc}
\usepackage[small]{caption}
\usepackage{graphicx}
\usepackage{amsmath}
\usepackage{booktabs}
\usepackage{algorithm}
\usepackage{algorithmic}
\urlstyle{same}

\title{Exploration d'un Resnet pour la classification d'images}

\author{
Nicolas Dépelteau
\and
Dung Nguyen
\affiliations
Polytechnique Montréal
\emails
nicolas.deplteau@polymtl.ca,
thi-ngoc-dung.nguyen@polymtl.ca

}
\begin{document}

\maketitle

\begin{abstract}
    Dans ce rapport écrit pour notre projet dans le cadre du cours INF8225,
    nous présentons les bénificents de l'utilisation
    d'un Resnet dans le cadre de la classification d'images. Nous avons également comparé les
    résultats de notre Resnet avec ceux d'un réseau de neurones convolutif simple. 
\end{abstract}

\section{Introduction}

TODO Une introduction avec un petit résumé des travaux antérieurs qui existent dans la littérature

\section{Les réseaux convolutifs}
Un réseau convolutif est un réseau dans lequel les neurones sont organisés en couches.
Chaque couche est composée de plusieurs neurones. Chaque neurone est connecté à tous les neurones de la couche précédente.
Chaque connexion est associée à un poids. Lorsque le réseau est activé, chaque neurone calcule une somme
pondérée des valeurs de sortie des neurones de la couche précédente.
Cette opération est une convolution dans le cas d'un réseau convolutif.
Cette somme est ensuite passée à une fonction d'activation.
La fonction d'activation est utilisée pour introduire une non-linéarité dans le réseau.
La fonction d'activation la plus utilisée est la fonction ReLU. La fonction ReLU est définie comme suit:
\begin{equation}
    f(x) = max(0, x)
\end{equation}

\subsection{Convolution}
L'opération de convolution est une opération mathématique qui est utilisée pour extraire des caractéristiques d'une image.
L'opération de convolution est définie comme suit:
\begin{equation}
    S(i, j) = (I * K)(i, j) = \sum_{m} \sum_{n} I(m, n)K(i - m, j - n)
\end{equation}
où $I$ est l'image d'entrée, $K$ est le noyau de convolution et $S$ est l'image de sortie.
Le noyau de convolution est une matrice de taille $m \times n$.
L'image de sortie est une image de taille $m \times n$.
L'image de sortie est calculée en faisant glisser le noyau de convolution sur l'image d'entrée.

\subsection{Pooling}
Le pooling est une opération qui est utilisée pour réduire la taille de la carte des caractéristiques.
Le pooling est effectué en faisant glisser une fenêtre sur la carte des caractéristiques.
La valeur de sortie de la fenêtre est la valeur maximale de la fenêtre dans le cas du max pooling.
Toutefois il existe aussi d'autres types de pooling comme le average pooling.
Le average pooling est une opération qui est similaire au max pooling.
La valeur de sortie de la fenêtre est la moyenne des valeurs de la fenêtre.
La taille de la fenêtre est un hyperparamètre du réseau.

\subsection{Le problème  de dégradation}

TODO l'expliquation du problème de dégradation

Le problème de dégradation est un problème qui est présent dans les réseaux profonds.
Ce problème est causé par la présence de couches profondes dans le réseau.




\section{Resnet}

TODO l'expliquation des Resnet

\section{Expérimentation}

TODO l'expliquation de l'expérimentation

\subsection{Le jeu de données}

TODO l'expliquation du jeu de données

\subsection{Les hyperparamètres}

TODO l'expliquation des hyperparamètres

\subsection{Les résultats}

TODO l'expliquation des résultats

\section{Analyse critique de l'approche}

TODO l'analyse critique de l'approche

\section{Conclusion}

TODO la conclusion


\bibliographystyle{named}
\bibliography{rapport}

TODO la bibliographie

\end{document}
